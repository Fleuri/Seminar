\documentclass{acm_proc_article-sp}
\begin{document}

\title{Using Software-defined Networking and Big Data for mutual performance benefits}

%
% You need the command \numberofauthors to handle the 'placement
% and alignment' of the authors beneath the title.
%
% For aesthetic reasons, we recommend 'three authors at a time'
% i.e. three 'name/affiliation blocks' be placed beneath the title.
%
% NOTE: You are NOT restricted in how many 'rows' of
% "name/affiliations" may appear. We just ask that you restrict
% the number of 'columns' to three.
%
% Because of the available 'opening page real-estate'
% we ask you to refrain from putting more than six authors
% (two rows with three columns) beneath the article title.
% More than six makes the first-page appear very cluttered indeed.
%
% Use the \alignauthor commands to handle the names
% and affiliations for an 'aesthetic maximum' of six authors.
% Add names, affiliations, addresses for
% the seventh etc. author(s) as the argument for the
% \additionalauthors command.
% These 'additional authors' will be output/set for you
% without further effort on your part as the last section in
% the body of your article BEFORE References or any Appendices.

\numberofauthors{1} %  in this sample file, there are a *total*
% of EIGHT authors. SIX appear on the 'first-page' (for formatting
% reasons) and the remaining two appear in the \additionalauthors section.
%
\author{
% You can go ahead and credit any number of authors here,
% e.g. one 'row of three' or two rows (consisting of one row of three
% and a second row of one, two or three).
%
% The command \alignauthor (no curly braces needed) should
% precede each author name, affiliation/snail-mail address and
% e-mail address. Additionally, tag each line of
% affiliation/address with \affaddr, and tag the
% e-mail address with \email.
%
% 1st. author
\alignauthor Lauri Suomalainen \\
\affaddr{University of Helsinki}\\
}       




\date{1 October 2014}
% Just remember to make sure that the TOTAL number of authors
% is the number that will appear on the first page PLUS the
% number that will appear in the \additionalauthors section.

\maketitle



\terms{Networking, Cloud computing.}

\keywords{Software-defined networking, Big Data} % NOT required for Proceedings

\section{Introduction}
Big Data and Software-Defined Networking are both relatively new technologies in the field of computer science. The advanced methods for data acquisition and the abundance of data itself allowed and caused the big data paradigm to emerge in the beginning of the 2000s. The most well known applications and algorithm for big data, namely Google's The Google File System (GFS) and MapReduce, debuted in 2003 and 2004 respectively \cite{Ghemawat:2003:GFS:1165389.945450,Dean:2008:MSD:1327452.1327492} and their success has since inspired other similar applications like Apache's open-source implementation Hadoop. \cite{Hadoop}.

Software-defined Networking (SDN) is newer both conceptionally and in practice. Network Function Virtualization (NFV) is also closely connected to SDN and both are often discussed in the same context. The first SDN applications, the POX Controller \cite{POX} and the OpenFlow network protocol \cite{McKeown-CCR2008}  were released in 2007 and 2008 and many other SDN and NFV related were soon to follow, most notably the massive company backed SDN project OpenDaylight \cite{ODL}. 

\section{Software-defined Networking}
\section{Big data challenges}
\section{How can SDN help?}
\section{SDN challenges?}
\section{How can Big Data help?}
\section{Conclusions}

\bibliographystyle{acm}
\bibliography{references}

\end{document}
