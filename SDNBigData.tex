\documentclass{acm_proc_article-sp}
\usepackage{url}
\usepackage{subfig}
\usepackage{rotating,tabularx}
\usepackage{float}
\usepackage{rotating}
\begin{document}

\title{Using Software-defined Networking and Big Data for mutual performance benefits}

%
% You need the command \numberofauthors to handle the 'placement
% and alignment' of the authors beneath the title.
%
% For aesthetic reasons, we recommend 'three authors at a time'
% i.e. three 'name/affiliation blocks' be placed beneath the title.
%
% NOTE: You are NOT restricted in how many 'rows' of
% "name/affiliations" may appear. We just ask that you restrict
% the number of 'columns' to three.
%
% Because of the available 'opening page real-estate'
% we ask you to refrain from putting more than six authors
% (two rows with three columns) beneath the article title.
% More than six makes the first-page appear very cluttered indeed.
%
% Use the \alignauthor commands to handle the names
% and affiliations for an 'aesthetic maximum' of six authors.
% Add names, affiliations, addresses for
% the seventh etc. author(s) as the argument for the
% \additionalauthors command.
% These 'additional authors' will be output/set for you
% without further effort on your part as the last section in
% the body of your article BEFORE References or any Appendices.

\numberofauthors{1} %  in this sample file, there are a *total*
% of EIGHT authors. SIX appear on the 'first-page' (for formatting
% reasons) and the remaining two appear in the \additionalauthors section.
%
\author{
% You can go ahead and credit any number of authors here,
% e.g. one 'row of three' or two rows (consisting of one row of three
% and a second row of one, two or three).
%
% The command \alignauthor (no curly braces needed) should
% precede each author name, affiliation/snail-mail address and
% e-mail address. Additionally, tag each line of
% affiliation/address with \affaddr, and tag the
% e-mail address with \email.
%
% 1st. author
\alignauthor Lauri Suomalainen \\
\affaddr{University of Helsinki}\\
}       




\date{1 October 2014}
% Just remember to make sure that the TOTAL number of authors
% is the number that will appear on the first page PLUS the
% number that will appear in the \additionalauthors section.

\maketitle



\terms{Networking, Cloud computing.}

\keywords{Software-defined networking, Big Data} % NOT required for Proceedings

\section{Introduction}
Big Data and Software-Defined Networking are both relatively new technologies in the field of computer science. The advanced methods for data acquisition and the abundance of data itself allowed and caused the big data paradigm to emerge in the beginning of the 2000s. The most well known applications and algorithm for big data, namely Google's The Google File System (GFS) and MapReduce, debuted in 2003 and 2004 respectively \cite{Ghemawat:2003:GFS:1165389.945450,Dean:2008:MSD:1327452.1327492} and their success has since inspired other similar applications like Apache's open-source implementation Hadoop \cite{Hadoop} and even paradigmatically alternative solutions such as SPARK \cite{Spark}.

Software-defined Networking (SDN) is newer both conceptionally and in practice. Network Function Virtualization (NFV) is also closely connected to SDN and both are often discussed in the same context. While many ideas that would form SDN have been evolving for over past twenty years \cite{Feamster:2013:RS:2559899.2560327}, one could argue that the explicit concept for SDN was first introduced in 2004 \cite{robert2012system}. The first SDN applications, the NOX Controller \cite{NOX} and the OpenFlow network protocol \cite{McKeown-CCR2008}  were released in 2008 and 2009 and many other SDN and NFV related implementations were soon to follow, most notably the massive company backed SDN project Open Daylight \cite{ODL}.

When it comes combining SDN and Big Data, the research and applications are not just relatively new but de facto recent. Many examples I am going to discuss are about two years old or newer. This also means, that the actual concrete research and experimentation is rather hard to come by and the few examples mostly revolve around optimising Big Data applications with SDN. However, the possibilities, opportunities and challenges of both SDN and Big Data have been and are discussed comprehensively. In this paper I am going to discuss both actual developed solutions as well as speculate the opportunities presented by both technologies to improve each others' performance and functionality. I will start by briefly introducing the key concepts and architectural design of the current general Software-defined Networking paradigm, continue by presenting the known challenges for Big Data that are relevant for SDN and for which SDN could provide solutions and then discuss the solutions, both actual and hypothetical, themselves. The other half of this paper will focus on improving and performing SDN with Big Data and will follow the same format of taking a look at the challenges and then speculating possible solutions. Finally I will try to draw some conclusions and consider the future possibilities of the two technologies. 

\section{Software-defined Networking}

Software-defined Networking is a networking paradigm and an architectural design. According to Feamster et al. the major design choices of SDN are the decoupling of the network control plane and data plane and centralisation of control so that a single program monitors and modifies all network elements \cite{Feamster:2013:RS:2559899.2560327}. In essence this adds a previously non-existent abstraction level in-between the physical network and network applications freeing the network programmer from having to concern him or herself with the actual logical implementations of network devices. This along with the Network Functions Virtualization allows for configuring networks and network functions easily with well defined APIs such as OpenFlow \cite{McKeown-CCR2008}. For example with Openflow switches, both physical and virtual, it is possible to use commodity hardware and servers for middlebox functions. Figure ~\ref{fig:architecture} shows the typical SDN controller architecture with OpenFlow enabled switches and the interfaces between the data plane and control plane, the South-bound interfaces, as well as the North-bound interface, usually an API, between the controller and network applications. Virtualization adds to SDN's benefits: Multiple networks can co-exist on the same hardware and to have their topology decoupled from the physical network topology. \cite{Azodolmolky} The physical network resource usage is also enhanced through network resource pooling. 

\begin{figure}[ht!]
\centering
\floatstyle{boxed}
\includegraphics[width=80mm]{"Controller architecture diagram".jpg}
\caption{General SDN controller architecture}
\label{fig:architecture}
\end{figure} 




\section{Big data challenges}



Massive Data challenges \cite{Frontiers} \cite{Azodolmolky}
Scaling of resources.
Optimization of infrastructure.

\section{How can SDN help?}

Scheduling. \cite{Scheduling}
Network optimization \cite{Wang:2012:PYN:2342441.2342462}
Other possible speculation

\section{SDN challenges?}

Security \cite{Kreutz13}
Horizontal Scalability (Big data unlikely to help, but worth mentioning)
Machine learning and network optimization according to collected data. Traffic pattern prediction. (Haven't searched for texts for this yet)

\section{How can Big Data help?}

Address the parts above.

\section{Conclusions}

\bibliographystyle{acm}
\bibliography{references}

\end{document}
