\documentclass{acm_proc_article-sp}
\usepackage{url}
\usepackage{subfig}
\usepackage{rotating,tabularx}
\usepackage{float}
\usepackage{rotating}
\begin{document}

\title{Using Software-defined Networking and Big Data for mutual performance benefits}

%
% You need the command \numberofauthors to handle the 'placement
% and alignment' of the authors beneath the title.
%
% For aesthetic reasons, we recommend 'three authors at a time'
% i.e. three 'name/affiliation blocks' be placed beneath the title.
%
% NOTE: You are NOT restricted in how many 'rows' of
% "name/affiliations" may appear. We just ask that you restrict
% the number of 'columns' to three.
%
% Because of the available 'opening page real-estate'
% we ask you to refrain from putting more than six authors
% (two rows with three columns) beneath the article title.
% More than six makes the first-page appear very cluttered indeed.
%
% Use the \alignauthor commands to handle the names
% and affiliations for an 'aesthetic maximum' of six authors.
% Add names, affiliations, addresses for
% the seventh etc. author(s) as the argument for the
% \additionalauthors command.
% These 'additional authors' will be output/set for you
% without further effort on your part as the last section in
% the body of your article BEFORE References or any Appendices.

\numberofauthors{1} %  in this sample file, there are a *total*
% of EIGHT authors. SIX appear on the 'first-page' (for formatting
% reasons) and the remaining two appear in the \additionalauthors section.
%
\author{
% You can go ahead and credit any number of authors here,
% e.g. one 'row of three' or two rows (consisting of one row of three
% and a second row of one, two or three).
%
% The command \alignauthor (no curly braces needed) should
% precede each author name, affiliation/snail-mail address and
% e-mail address. Additionally, tag each line of
% affiliation/address with \affaddr, and tag the
% e-mail address with \email.
%
% 1st. author
\alignauthor Lauri Suomalainen \\
\affaddr{University of Helsinki}\\
}       




\date{1 October 2014}
% Just remember to make sure that the TOTAL number of authors
% is the number that will appear on the first page PLUS the
% number that will appear in the \additionalauthors section.

\maketitle



\terms{Networking, Cloud computing.}

\keywords{Software-defined networking, Big Data} % NOT required for Proceedings

\section{Introduction}
Big Data and Software-Defined Networking are both relatively new technologies in the field of computer science. The advanced methods for data acquisition and the abundance of data itself allowed and caused the big data paradigm to emerge in the beginning of the 2000s. The most well known applications and algorithm for big data, namely Google's The Google File System (GFS) and MapReduce, debuted in 2003 and 2004 respectively \cite{Ghemawat:2003:GFS:1165389.945450,Dean:2008:MSD:1327452.1327492} and their success has since inspired other similar applications like Apache's open-source implementation Hadoop \cite{Hadoop} and even paradigmatically alternative solutions such as SPARK \cite{Spark}.

Software-defined Networking (SDN) is newer both conceptionally and in practice. Network Function Virtualization (NFV) is also closely connected to SDN and both are often discussed in the same context. While many ideas that would form SDN have been evolving for over past twenty years \cite{Feamster:2013:RS:2559899.2560327}, one could argue that the explicit concept for SDN was first introduced in 2004 \cite{robert2012system}. The first SDN applications, the NOX Controller \cite{NOX} and the OpenFlow network protocol \cite{McKeown-CCR2008}  were released in 2008 and 2009 and many other SDN and NFV related implementations were soon to follow, most notably the massive company backed SDN project Open Daylight \cite{ODL}.

When it comes combining SDN and Big Data, the research and applications are not just relatively new but de facto recent. Many examples I am going to discuss are about two years old or newer. This also means, that the actual concrete research and experimentation is rather hard to come by and the few examples mostly revolve around optimising Big Data applications with SDN. However, the possibilities, opportunities and challenges of both SDN and Big Data have been and are discussed comprehensively. In this paper I am going to discuss both actual developed solutions as well as speculate the opportunities presented by both technologies to improve each others' performance and functionality. I will start by briefly introducing the key concepts and architectural design of the current general Software-defined Networking paradigm, continue by presenting the known challenges for Big Data that are relevant for SDN and for which SDN could provide solutions and then discuss the solutions, both actual and hypothetical, themselves. The other half of this paper will focus on improving and performing SDN with Big Data and will follow the same format of taking a look at the challenges and then speculating possible solutions. Finally I will try to draw some conclusions and consider the future possibilities of the two technologies. 

\section{Software-defined Networking}

Software-defined Networking is a networking paradigm and an architectural design. According to Feamster et al. the major design choices of SDN are the decoupling of the network control plane and data plane and centralisation of control so that a single program monitors and modifies all network elements \cite{Feamster:2013:RS:2559899.2560327}. In essence this adds a previously non-existent abstraction level in-between the physical network and network applications freeing the network programmer from having to concern him or herself with the actual logical implementations of network devices. This along with the Network Functions Virtualization allows for configuring networks and network functions easily with well defined APIs such as OpenFlow \cite{McKeown-CCR2008}. For example with Openflow switches, both physical and virtual, it is possible to use commodity hardware and servers for middlebox functions. Figure ~\ref{fig:architecture} shows the typical SDN controller architecture with OpenFlow enabled switches and the interfaces between the data plane and control plane, the South-bound interfaces, as well as the North-bound interface, usually an API, between the controller and network applications. Virtualization adds to SDN's benefits: Multiple networks can co-exist on the same hardware and to have their topology decoupled from the physical network topology \cite{Azodolmolky}. The physical network resource usage is also enhanced through network resource pooling. 

\begin{figure}[ht!]
\centering
\floatstyle{boxed}
\includegraphics[width=90mm]{"Controller architecture diagram".jpg}
\caption{General SDN controller architecture}
\label{fig:architecture}
\end{figure} 




\section{Big data challenges}

In their article \textit{Cloud Computing Networking: Challenges and Opportunities for Innovations} \cite{Azodolmolky} Siamak Azodolmolky et al. investigate the known challenges Cloud Computing faces. While the challenges are discussed within a purview of general cloud computing, many of them apply directly or partly to Big Data as well. These are some of the challenges that could be addressed with SDN to some degree.

Application performance is always an issue. This aspect is especially emphasized in a multi-tenant cloud with multiple applications running. The cloud service should guarantee users the bandwidth they need as defined in cloud's service level agreements. This is also true for Big Data clusters with many users and simultaneous jobs.

Azodolmolky et al. consider the complexities and challenges multiple forwarding policies bring. According to them the policies directly impact the configurations of network devices and with vendors having protocols specific for their products complicate the building, maintaining and operating of a Cloud network.

Data centers often have a network topology designed to accomodate the needs of the application and network traffic. This also means, that physically networks are very static and changing the topology to respond e.g. changing traffic patterns or application requirements is a labourious and time-consuming effort requiring manual configuration of switches, routers and other network devices.

Network appliances and servers are tied to physical network. Naturally this means that cloud services are dependent on location. Azodolmolky et al. say, that the way IP addresses are typically determined by VLAN or the subnets and how they are based on physical switch port configuration complicate virtual machine migrations in the network. This handicaps dynamic resource utilization and makes resource allocation inflexible. This challenge is very relevant to Big Data computation as the computational demands in a Big Data system or a Cloud in general are variable and the scaling and optimized usage of computational resources is necessary \cite{Frontiers} in many aspects, such as energy consumption and upkeep costs.

\section{How can SDN help?}


Software-defined Networking and Network function virtualization have much to offer for Big Data and Cloud computing in general. With the abstraction levels SDN offers, construction of suitable networks for Big Data application becomes less challenging and with the aid of virtualization the network functions can be run on commodity hardware, such as ordinary x86 servers. The possibility to not use specialized hardware can be a huge benefit both technically and financially. Specialized hardware is expensive and the configuration possibilities and ways differ from one vendor to another. OpenFlow-enabled switches can be used to run middlebox functions and thus the unified programming and configuration interface allows for easier operation . This would also help in situation in which the physical device fails and it is required to move its functions to another device. In such situation no additional configuration would be required even if the device would differ from the original one. 

In their paper \textit{Bandwidth-Aware Scheduling with SDN in Hadoop: A New Trend for Big Data} \cite{Scheduling} Peng Qin et al. from Huazong University of Science and Technology note the similarities between the organization and architecture of a Hadoop cluster and an OpenFlow SDN network. The architectural similarities are shown in figure ~\ref{fig:hadoop}.

\begin{figure}[ht!]
\centering
\floatstyle{boxed}
\includegraphics[width=90mm]{"Hadoop".png}
\caption{Architecture of Hadoop Big Data Processing with SDN according to Qin et al. \cite{Scheduling}}
\label{fig:hadoop}
\end{figure} 

Qin et. al argue, that Hadoop job scheduling could be significantly improved with SDN. Efficient scheduling is essential for any Big Data application's entire performance. Scheduling ensures that the workloads are assigned evenly and efficiently among the computing nodes in the system. One of the aspects that makes scheduling in Hadoop particularly interesting is the fact that it is proven an NP-complete problem \cite{Fischer:2010:ATE:1810479.1810484} and thus there is no feasible algorithm for finding optimal Hadoop task assignments. Therefore the algorithms have to rely on heuristics. The bandwidth in a Big Data cluster is a scarce resource and that is why schedulers try to impose data locality as much as possible so that links between nodes do not get congested and the overhead caused by data movement is minimal. According to Qin et al. many methodologies aiming to enhance data locality have been proposed. Hadoop Default Scheduler (HDS) and BAlance-Reduce scheduler (BAR) are discussed in their paper. The major shortcoming with these methods however is the fact that they do not allocate tasks in a global view or ignore the available bandwidth whilst making the allocation decision.

Qin et al. propose a solution that exploits SDN's capabilities to monitor network traffic in real time to obtain available link bandwidth and then use it as a variable for computing the allocation decisions. The scheduler is called Bandwidth-Aware Scheduling with Sdn in hadoop or BASS in short. To developers' knowledge this is the first Hadoop scheduling solution utilizing SDN. BASS works by allocating bandwidth in a time slot manner and using that information a task is assigned either locally or remotely to computation nodes depending on which optione yields a shorter completion time for the task. Developers' test results show BASS outperforming both HDS and BAR consistently: With a sorting job with 5 gigabytes of data BASS could complete the job in 1572 seconds whereas BAR completed it in 1632 seconds and HDS in 1859 seconds. An interesting notion is that in this case BASS handles almost 5 \% less data locally than BAR but produces about 4 \% better throughput. One other benefit that BASS provides is related to Hadoop's overall architecture. Hadoop assumes that all cluster nodes are dedicated to a single user and because of that the performance is likely to deteriorate in a shared environment. With BASS however, the network is presented in a global view and thus the bandwidth is a shared resource and known to all users. BASS handling the bandwidth in real time and as a common resource pool implicitly allocates nodes for different tasks fairly.

In their work \textit{Programming Your Network at Run-time for Big Data Applications} \cite{Wang:2012:PYN:2342441.2342462} Guohui Wang et al. speculate the possibility to utilize SDN along with optical switches to configure a Big Data network at runtime to meet the requirements of changing traffic patterns. As mentioned earlier, many Big Data applications like Hadoop have a centralized management structure which makes it possible to get application-level information and utilize that for network optimization. The writers note that their proposed scheme imposes certain challenges to SDN: In comparison to common use cases such as WAN traffic engineering or cloud network provisioning, configuring networks for big data jobs requires faster and more frequent flow table updates and that imposes scalability and update speed requirements for an SDN controller. In their setup the SDN controller is interfaced to the big data application's master node which manages the incoming job requests. The controller provides a general interface for device configuration, forwarding control and network state queries. For big data applications, the controller provides an interface for accepting traffic demand matrices from application controller. In this way application controllers can report traffic demands and traffic structure from jobs and issue a network command to set up the topology to address the needs. Wang et al. similarly to Qin et al.\cite{Scheduling} also note the possibility for application controllers to use network information SDN controller provides to make better decisions  with scheduling and job placement. The proposition aims to fix the shortcomings of current approaches. Many earlier approaches use only network level statistics but according to writers the performance can be poor without application-level view of traffic demands. This is because of the actual traffic demand is hard to estimate with only network-level statistics. Also the network optimization without considering the structure of an application could cause blocking among interdependent applications and thus poor performance. 

Wang et al. also suggest that SDN could be used for flow-level traffic engineering \cite{Wang:2012:PYN:2342441.2342462}. The traffic demand and structural pattern  allows splitting and re-routing of management and data flows on different routes, though Wang et al. warn that the overhead for setting flow control rules may negate the benefits traffic engineering could offer.

\section{SDN challenges}

Software-defined networking is a relatively new concept and technology that has sparked a lot of interest both in scientific community as well as in the enterprise world. Even though the developmet in the field is and has been rapid, SDN still has significant challenges that have prevented its wide adoption to large-scale business environment. Sezer et al. list the key challenges to be managing the trade-off between performance and flexibility, network scalability because centralization of control in software-defined networks imposes a limit to network size, network security and interoperability between traditional and software-defined networks \cite{sezer2013we}. Of these Big Data could clearly be beneficial for SDN security.

Kreutz et al. argue, that while SDN provides new possibilities for solving common networking problems and enables introduction of even more sophisticated policies, like security and dependability, the discussion of SDN has revolved around architectural solutions and security topics of SDN itself have been neglected  \cite{Kreutz13}. The programmability of the network and centralization of control, while being the basis of the whole SDN paradigm and architecture, create new fault and attack planes. Despite of that, Kreutz et al. argue that SDN is by no means inherently less secure than traditional networks, but the threats are of different nature and thus have to be dealt with differently.
Kreutz et al. list seven different threat vectors that apply to software defined networking.

\begin{itemize}
\item \textbf{Forged or faked traffic flows:} These could be used to launch Denial of Service attacks against switches or even the controller.
\item \textbf{Attacks on vulnerabilities of the switches:} An attacker gaining control of a switch could use it to drop or slow down packets, clone or reroute them maliciously or inject traffic and forged requests in purpose of overloading and disrupting the controller and other switches in the network.
\end{itemize}

Machine learning and network optimization according to collected data. Traffic pattern prediction. (Haven't searched for texts for this yet)

\section{How can Big Data help?}

Address the parts above.

\section{Conclusions}

\bibliographystyle{acm}
\bibliography{references}

\end{document}
