\documentclass{acm_proc_article-sp}
\usepackage{url}
\usepackage{subfig}
\usepackage{rotating,tabularx}
\usepackage{float}
\usepackage{rotating}
\begin{document}

\title{Using Software-defined Networking and Big Data for mutual performance benefits}

%
% You need the command \numberofauthors to handle the 'placement
% and alignment' of the authors beneath the title.
%
% For aesthetic reasons, we recommend 'three authors at a time'
% i.e. three 'name/affiliation blocks' be placed beneath the title.
%
% NOTE: You are NOT restricted in how many 'rows' of
% "name/affiliations" may appear. We just ask that you restrict
% the number of 'columns' to three.
%
% Because of the available 'opening page real-estate'
% we ask you to refrain from putting more than six authors
% (two rows with three columns) beneath the article title.
% More than six makes the first-page appear very cluttered indeed.
%
% Use the \alignauthor commands to handle the names
% and affiliations for an 'aesthetic maximum' of six authors.
% Add names, affiliations, addresses for
% the seventh etc. author(s) as the argument for the
% \additionalauthors command.
% These 'additional authors' will be output/set for you
% without further effort on your part as the last section in
% the body of your article BEFORE References or any Appendices.

\numberofauthors{1} %  in this sample file, there are a *total*
% of EIGHT authors. SIX appear on the 'first-page' (for formatting
% reasons) and the remaining two appear in the \additionalauthors section.
%
\author{
% You can go ahead and credit any number of authors here,
% e.g. one 'row of three' or two rows (consisting of one row of three
% and a second row of one, two or three).
%
% The command \alignauthor (no curly braces needed) should
% precede each author name, affiliation/snail-mail address and
% e-mail address. Additionally, tag each line of
% affiliation/address with \affaddr, and tag the
% e-mail address with \email.
%
% 1st. author
\alignauthor Lauri Suomalainen \\
\affaddr{University of Helsinki}\\
}       




\date{1 October 2014}
% Just remember to make sure that the TOTAL number of authors
% is the number that will appear on the first page PLUS the
% number that will appear in the \additionalauthors section.

\maketitle

\begin{abstract}

Big Data is an umbrella term for technologies designed to handle massive data sets that traditional  data processing techniques could not or have serious difficulties in doing so. Software-defined Networking is a technology and a paradigm aiming to abstract network configuration and thus improve network usage and network application development. Both are somewhat new concepts that have gained public recognition in the beginning of the 21st century. The technologies have mostly been researched and developed separately, but there have been proposals recently for combining the two. In this paper I review the challenges Big Data and Software-defined Networking face and possible benefits they could offer to each other.

\end{abstract}



\terms{Networking, Cloud computing.}

\keywords{Software-defined networking, Big Data} % NOT required for Proceedings

\section{Introduction}
Big Data and Software-Defined Networking are both relatively new technologies in the field of computer science. The advanced methods for data acquisition and the abundance of data itself allowed and caused the big data paradigm to emerge in the beginning of the 2000s. The most well known applications and algorithm for big data, namely Google's The Google File System (GFS) and MapReduce, debuted in 2003 and 2004 respectively \cite{Ghemawat:2003:GFS:1165389.945450,Dean:2008:MSD:1327452.1327492} and their success has since inspired other similar applications like Apache's open-source implementation Hadoop \cite{Hadoop} and even paradigmatically alternative solutions such as Spark \cite{Spark}.

Software-defined Networking (SDN) is newer both conceptionally and in practice. Network Function Virtualisation (NFV) is also closely connected to SDN and both are often discussed in the same context. I will introduce these concepts briefly in the next section.

While many ideas that would form SDN have been evolving for over past twenty years \cite{Feamster:2013:RS:2559899.2560327}, one could argue that the explicit concept for SDN was first introduced in 2004 \cite{robert2012system}. The first SDN applications, the NOX Controller \cite{NOX} and the OpenFlow network protocol \cite{McKeown-CCR2008}  were released in 2008 and 2009 and many other SDN and NFV related implementations were soon to follow, most notably the massive company backed SDN project Open Daylight \cite{ODL}.

When it comes combining SDN and Big Data, the research and applications are not just relatively new but de facto recent. Many examples I am going to discuss are about two years old or newer. This also means, that the actual concrete research and experimentation is rather hard to come by and the few examples mostly revolve around optimising Big Data applications with SDN. However, the possibilities, opportunities and challenges of both SDN and Big Data have been and are discussed comprehensively. In this paper I am going to discuss both actual developed solutions as well as speculate the opportunities presented by both technologies to improve each others' performance and functionality. I will start by briefly introducing Big data briefly in section 2 and the key concepts and architectural design of the current general Software-defined Networking paradigm in section 3. In section 4 I will continue by presenting the known challenges for Big Data that are relevant for SDN and for which SDN could provide solutions. In section 5 I will then discuss the solutions, both actual and hypothetical, themselves. The other half of this paper is similar in structure, but with reversed roles. Section 6 will outline some key challenges that could possibly be addressed with Big Data applications. Section 7 discusses the possible Big Data solution proposals. I will follow with a short summary, conclusions and reflections in section 8.  

\section{Big Data}

Big Data refers to analysis and extraction of meaningful data from massive data sets. The nature of these data sets are commonly referred with 'three V's' \cite{Benjamins:2014:BDH:2611040.2611042}

\begin{itemize}
\item \textbf{Volume: } The data sets are typically massive.
\item \textbf{Variety: } The data in sets is very heterogeneous and often unstructured
\item \textbf{Velocity: } The data should be handled in a reasonable time. In some cases it is vital to extract meaningful data in real-time. One example could be the sensor data collected by an airliner's sensor systems. The data is needed in real time for the safe control of the plane.
\end{itemize}

Specific Big Data technologies are needed, because this type of data can not be handled by more traditional approaches well enough. Typical approach for solving this problem is with powerful distributed parallel computational clusters and 'divide and conquer' algorithms, such as Google's MapReduce \cite{Dean:2008:MSD:1327452.1327492} and the open-source approach Hadoop \cite{Hadoop}.

\section{Software-defined Networking}

Software-defined Networking is a networking paradigm and an architectural design. According to Feamster et al. the major design choices of SDN are the decoupling of the network control plane and data plane and centralisation of control so that a single program monitors and modifies all network elements \cite{Feamster:2013:RS:2559899.2560327}. In essence this adds a previously non-existent abstraction level in-between the physical network and network applications freeing the network programmer from having to concern him or herself with the actual logical implementations of network devices. This along with the Network Functions Virtualisation (NFV) allows for configuring networks and network functions easily with well defined APIs such as OpenFlow \cite{McKeown-CCR2008}. Put simply, Network Functions Virtualisation is network architecture concept that aims to decouple the network node functionality from the actual physical devices. This is commonly done with virtual machines. For example with Openflow switches, both physical and virtual, it is possible to use commodity hardware and servers for middle box functions. Figure ~\ref{fig:architecture} shows the typical SDN controller architecture with OpenFlow enabled switches and the interfaces between the data plane and control plane, the South-bound interfaces, as well as the North-bound interface, usually an API, between the controller and network applications. Virtualisation adds to SDN's benefits: Multiple networks can co-exist on the same hardware and to have their topology decoupled from the physical network topology \cite{Azodolmolky}. The physical network resource usage is also enhanced through network resource pooling. 

\begin{figure}[ht!]
\centering
\floatstyle{boxed}
\includegraphics[width=90mm]{"Controller architecture diagram".jpg}
\caption{General SDN controller architecture}
\label{fig:architecture}
\end{figure} 




\section{Big data challenges}

In their article \textit{Cloud Computing Networking: Challenges and Opportunities for Innovations} \cite{Azodolmolky} Siamak Azodolmolky et al. investigate the known challenges Cloud Computing faces. While the challenges are discussed within a purview of general cloud computing, many of them apply directly or partly to Big Data as well. These are some of the challenges that could be addressed with SDN to some degree.

Application performance is always an issue. This aspect is especially emphasised in a multi-tenant cloud with multiple applications running. The cloud service should guarantee users the bandwidth they need as defined in cloud's service level agreements. This is also true for Big Data clusters with many users and simultaneous jobs.

Azodolmolky et al. consider the complexities and challenges multiple forwarding policies bring \cite{Azodolmolky}. According to them the policies directly impact the configurations of network devices and with vendors having protocols specific for their products complicate the building, maintaining and operating of a Cloud network.

Data centres often have a network topology designed to accommodate the needs of the application and network traffic. This also means, that physically networks are very static and changing the topology to respond e.g. changing traffic patterns or application requirements is a labourious and time-consuming effort requiring manual configuration of switches, routers and other network devices.

Network appliances and servers are tied to physical network. Naturally this means that cloud services are dependent on location. Azodolmolky et al. say, that the way IP addresses are typically determined by VLAN or the subnets and how they are based on physical switch port configuration complicate virtual machine migrations in the network. This handicaps dynamic resource utilisation and makes resource allocation inflexible. This challenge is very relevant to Big Data computation as the computational demands in a Big Data system or a Cloud in general are variable and the scaling and optimised usage of computational resources is necessary \cite{Frontiers} in many aspects, such as energy consumption and upkeep costs.

\section{How can SDN help}


Software-defined Networking and Network function virtualisation have much to offer for Big Data and Cloud computing in general. With the abstraction levels SDN offers, construction of suitable networks for Big Data application becomes less challenging and with the aid of virtualisation the network functions can be run on commodity hardware, such as ordinary x86 servers. The possibility to not use specialised hardware can be a huge benefit both technically and financially. Specialised hardware is expensive and the configuration possibilities and ways differ from one vendor to another. OpenFlow-enabled switches can be used to run middle box functions and thus the unified programming and configuration interface allows for easier operation . This would also help in situation in which the physical device fails and it is required to move its functions to another device. In such situation no additional configuration would be required even if the device would differ from the original one. 

In their paper \textit{Bandwidth-Aware Scheduling with SDN in Hadoop: A New Trend for Big Data} \cite{Scheduling} Peng Qin et al. from Huazong University of Science and Technology note the similarities between the organisation and architecture of a Hadoop cluster and an OpenFlow SDN network. The architectural similarities are shown in figure ~\ref{fig:hadoop}.

\begin{figure}[ht!]
\centering
\floatstyle{boxed}
\includegraphics[width=90mm]{"Hadoop".png}
\caption{Architecture of Hadoop Big Data Processing with SDN according to Qin et al. \cite{Scheduling}}
\label{fig:hadoop}
\end{figure} 

Qin et. al argue, that Hadoop job scheduling could be significantly improved with SDN. Efficient scheduling is essential for any Big Data application's entire performance. Scheduling ensures that the workloads are assigned evenly and efficiently among the computing nodes in the system. One of the aspects that makes scheduling in Hadoop particularly interesting is the fact that it is proven an NP-complete problem \cite{Fischer:2010:ATE:1810479.1810484} and thus there is no feasible algorithm for finding optimal Hadoop task assignments. Therefore the algorithms have to rely on heuristics. The bandwidth in a Big Data cluster is a scarce resource and that is why schedulers try to impose data locality as much as possible so that links between nodes do not get congested and the overhead caused by data movement is minimal. According to Qin et al. many methodologies aiming to enhance data locality have been proposed. Hadoop Default Scheduler (HDS) and BAlance-Reduce scheduler (BAR) are discussed in their paper. The major shortcoming with these methods however is the fact that they do not allocate tasks in a global view or ignore the available bandwidth whilst making the allocation decision.

Qin et al. propose a solution that exploits SDN's capabilities to monitor network traffic in real time to obtain available link bandwidth and then use it as a variable for computing the allocation decisions. The scheduler is called Bandwidth-Aware Scheduling with Sdn in hadoop [sic] or BASS in short. To developers' knowledge this is the first Hadoop scheduling solution utilising SDN. BASS works by allocating bandwidth in a time slot manner and using that information a task is assigned either locally or remotely to computation nodes depending on which option yields a shorter completion time for the task. Developers' test results show BASS outperforming both HDS and BAR consistently: With a sorting job with 5 gigabytes of data BASS could complete the job in 1572 seconds whereas BAR completed it in 1632 seconds and HDS in 1859 seconds. An interesting notion is that in this case BASS handles almost 5 \% less data locally than BAR but produces about 4 \% better throughput. One other benefit that BASS provides is related to Hadoop's overall architecture. Hadoop assumes that all cluster nodes are dedicated to a single user and because of that the performance is likely to deteriorate in a shared environment. With BASS however, the network is presented in a global view and thus the bandwidth is a shared resource and known to all users. BASS handling the bandwidth in real time and as a common resource pool implicitly allocates nodes for different tasks fairly.

In their work \textit{Programming Your Network at Run-time for Big Data Applications} \cite{Wang:2012:PYN:2342441.2342462} Guohui Wang et al. speculate the possibility to utilise SDN along with optical switches to configure a Big Data network at runtime to meet the requirements of changing traffic patterns. As mentioned earlier, many Big Data applications like Hadoop have a centralised management structure which makes it possible to get application-level information and utilise that for network optimisation. The writers note that their proposed scheme imposes certain challenges to SDN: In comparison to common use cases such as WAN traffic engineering or cloud network provisioning, configuring networks for big data jobs requires faster and more frequent flow table updates and that imposes scalability and update speed requirements for an SDN controller. In their set-up the SDN controller is interfaced to the big data application's master node which manages the incoming job requests. The controller provides a general interface for device configuration, forwarding control and network state queries. For big data applications, the controller provides an interface for accepting traffic demand matrices from application controller. In this way application controllers can report traffic demands and traffic structure from jobs and issue a network command to set up the topology to address the needs. Wang et al. similarly to Qin et al.\cite{Scheduling} also note the possibility for application controllers to use network information SDN controller provides to make better decisions  with scheduling and job placement. The proposition aims to fix the shortcomings of current approaches. Many earlier approaches use only network level statistics but according to writers the performance can be poor without application-level view of traffic demands. This is because of the actual traffic demand is hard to estimate with only network-level statistics. Also the network optimisation without considering the structure of an application could cause blocking among interdependent applications and thus poor performance. 

Wang et al. also suggest that SDN could be used for flow-level traffic engineering \cite{Wang:2012:PYN:2342441.2342462}. The traffic demand and structural pattern  allows splitting and re-routing of management and data flows on different routes, though Wang et al. warn that the overhead for setting flow control rules may negate the benefits traffic engineering could offer.

\section{SDN challenges}

Software-defined networking is a relatively new concept and technology that has sparked a lot of interest both in scientific community as well as in the enterprise world. Even though the development in the field is and has been rapid, SDN still has significant challenges that have prevented its wide adoption to large-scale business environment. Sezer et al. list the key challenges to be managing the trade-off between performance and flexibility, horizontal scalability in Software-defined Networks because the centralisation of control in software-defined networks imposes a limit to network size, network security and interoperability between traditional and software-defined networks \cite{sezer2013we}. When it comes to these challenges, Big Data applications could be clearly beneficial for the security in Software-defined networks.

Kreutz et al. argue, that while SDN provides new possibilities for solving common networking problems and enables introduction of even more sophisticated policies, like security and dependability, the discussion of SDN has revolved around architectural solutions and security topics of SDN itself have been neglected  \cite{Kreutz13}. The programmability of the network and centralisation of control, while being the basis of the whole SDN paradigm and architecture, create new fault and attack planes. Despite of that, Kreutz et al. argue that SDN is by no means inherently less secure than traditional networks, but the threats are of different nature and thus have to be dealt with differently.
Kreutz et al. list seven different threat vectors that apply to software defined networking. Some of these could have Big Data related solutions to them.

\begin{itemize}
\item \textbf{Forged or faked traffic flows:} These could be used to launch Denial of Service attacks against switches or even the controller.
\item \textbf{Attacks on vulnerabilities of the switches:} An attacker gaining control of a switch could use it to drop or slow down packets, clone or reroute them maliciously or inject traffic and forged requests in purpose of overloading and disrupting the controller and other switches in the network.
\item \textbf{Attacks on control plane communications} can be used to generate DoS attacks or for data theft.
\item \textbf{Attacks on vulnerabilities in controllers} are probably the most significant threats to Software-defined Networks. A faulty or malicious controller could potentially compromise the whole network.
\item \textbf{Lack of mechanisms to ensure trust between the controller and management applications.} SDN implementations still lack autonomic trust management for making sure that the application is trusted during its lifetime.
\item \textbf{Lack of trusted resources for forensics and remediation.} Current SDN implementations would require problem detection and both fast and secure recovery ability from incidents. Data has to retain its integrity and authenticity et cetera and good recovery requires safe and reliable system snapshots.
\end{itemize}

\section{How can Big Data help}

The discussion about Big Data empowering SDN is very recent and problematic in a sense that no peer-reviewed scientific texts are available at the time of writing this. However, numerous networking and SDN-related sites and professional technology blogs have speculated the possibilities and benefits Big Data could bring to SDN.

One of the main benefits of Software-defined Networking is, as the name suggests, the programmability of the network and especially at run-time. Michael Bushong from Plexxi Incorporated, a networking company with a strong focus on Software-defined networking, discusses the matter in a couple of articles \cite{Bushong2013, Bushong2013-2}. With network monitoring capabilities of SDN it is possible for a large network to generate similarly large volumes of data, such as infrastructure states and traffic, application and end-user data \cite{Bushong2013}. Theoretically the generated data could, with the aid of Big Data applications, be used to tune and configure the network and routing in real time automatically or the configuration could be computer assisted with a human network administrator making the final decisions. I argue, that data collected during a longer span of time could be used for predicting network requirements and end-user behaviour. For example analytics could be utilised for finding patterns in network: When does the network experience heavy traffic load and when are the network resources underutilised? Do certain real world events, like holidays or a company's product release, affect the amount and type of traffic a network experiences? The found patterns could then be used for proactive configuration of networks: When heavy traffic is expected, the SDN controller could fire up several new servers and tune the network topology accordingly. After the rush hour has passed, the network resources could be scaled down to avoid resource underutilisation.

In their whitepaper \textit{Big Data Analytics for United Security \textendash What Advantages Does an Agile Network Bring? (Issue 2)} \cite{Liu2014} Liu Swift from Huawei discusses possibilities of using Big Data analytics for network security. With the centralised control of SDN a controller can supervise the security of the entire network. Huawei has integrated behaviour analysis software with their controller. It searches logs, records security events in the entire network and analyses and discovers threats that could not be discovered by 'single-point defence measures'. The controller generates alarms which in turn can be handled by the network administrator or predefined policies: Swift explains, that suspicious traffic could be for example imported to a separate security center where applications and potential threats could be ran in a simulated sandbox-environment to observe their behaviour and gather security data without risking the actual network. With the gathered data a Big data analytics application could be used to generate and implement appropriate network policies in real time and even find previously unknown potential security risks. This is a huge improvement. Comparing to how a network administrator can only detect a security event through a manual post-event review and in a large network going through all the logs in a search for correlations is an arduous task. As a conclusion, Swift predicts automated network management to improve efficiency and reduce IT costs. 

\section{Conclusions}

The relevant challenges Big Data and Software-defined Networking face and the possible benefits and solutions they could offer to each other are listed on figure ~\ref{fig:summary}.

\begin{figure}[ht]
\centering
\floatstyle{boxed}
\includegraphics[width=90mm]{"summary".png}
\caption{Summary of the challenges and solutions}
\label{fig:summary}
\end{figure} 


Big data and Software-defined networking are both relatively new technologies and the latter is still waiting for its widespread applications. The research of combining both is extremely recent. Both technologies however show great promise separately as well as together. If Big Data and Software-defined networking manage to live up to their expectations in the future, the development could open vast possibilities. Apart from the obvious improvements in the Big Data application efficiency and infrastructural flexibility, and Software-defined Network automation and security, the technologies pave way for other sophisticated technological concepts. Combined with machine learning SDN coupled with Big Data could be used for building powerful cognitive networks that learn from their network events and adapt to their usage. The looming emergence of the so-called Internet of Things may be an opportunity for Big Data and SDN. I would argue that the sheer volume and heterogeneity of new endpoint devices introduced to networks will require some sort of a Big Data and SDN contraption to handle all the new data and network elements in a reasonable manner.

The research on combining Big Data and SDN is still nascent. Writing this paper at this time is both a hindrance and a benefit: The actual concrete research on matter is scarce, but speculation is abundant. It would be interesting to revisit this paper in few years to see, if the research takes off and whether the visions of future are realised or not. For the time being, it can be concluded that the mutual benefits of SDN and Big Data have a lot of potential, but the actual practical applications are yet to come and depend on the separate and joint advancements of both technologies.

\bibliographystyle{acm}
\bibliography{references}

\end{document}
